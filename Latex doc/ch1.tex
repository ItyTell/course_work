\chapter{SIR модель}
Є декілька найбільш розповсюджених математичних моделей для моделювання епідемій (та загалом інфекційних захворювань).
Спочатку розглянемо найрозповсюджені з них, а вже в другому розділі
створимо свою модель для COVID-19.

\section{SIR модель}
\par Одною з найпростіших моделей є SIR модель. 
В ній всі люди розглядаються у вигляді трьох груп: здорові (susceptible), хворі (infectious) та ті, що одужали (removed). 
Форсування людей між цими групами відбувається зі здорових у хворі 
і з хворих до одужаних.(див рис. \ref{SIR graph})  

\vspace{1cm}
\begin{risunok}[ht]
    \centering
    \begin{tikzpicture}
        \node
        [style={rectangle, draw=green!60, fill=green!5, very thick}]
        (Susceptible){Susceptible}; 
        \node
        [style={rectangle, draw=red!60, fill=red!5, very thick}]
        (Infectious)[right=of Susceptible]{Infectious};
        \node
        [style={rectangle, draw=blue!60, fill=blue!5, very thick}]
        (Removed)[right=of Infectious]{Removed};
        \draw[->, style={very thick}] (Susceptible.east)
        to node[above] {$ \frac{\beta}{N} $} (Infectious.west);
        \draw[->, style={very thick}] (Infectious.east)
        to node[above] {$  \gamma $} (Removed.west);
    \end{tikzpicture}
    \vspace{0.5cm}
    \caption{SIR model}
    \label{SIR graph}
\end{risunok}

Зв’язок між цими групами записують за допомогою системи диференціальних рівняннь (див рівняння \ref{SIR}).

\begin{equation}
    \begin{cases}
        \frac{dS}{dt} = - \frac{\beta}{N}SI          \\
        \frac{dI}{dt} = \frac{\beta}{N}SI - \gamma I \\
        \frac{dR}{dt} = \gamma I
    \end{cases}
\end{equation}
\label{SIR}


Перше рівняння описує,
що кількість здорових зменшується на величину 
пропорційну добутку хворих на здорових - кількість можливих контактів. 
Друге рівняння показує, що кількість хворих зростає на ту саму величну і зменшується на величину пропорційну до кількості хворих (кількість одужаних). 
Третє рівняння зібльшується на ту саму кількість одужаних\cite{salimipour_sir_2023}.
\par 
Додавши всі рівняння системи отримаємо - $ \frac{dS}{dt} + \frac{dI}{dt} + \frac{dR}{dt} = 0 $. 
Тож населення у нашій моделі не змінюєтся з часом: $N = S + I + R = const$.


\section{SEIR модель}

Часто хвороба характеризується не лише станами - хворий, здоровий, перехворівший.
Для подібних випадків додають додаткові стани.
Наприклад, якщо додати у SIR модель ще одну групу «заражені, але ще не взмозі інфікувати інших» то вийде SEIR модель. 
Тоді у порівнянні з SIR моделлю у схему додасться проміжний етап. (див рис. \ref{SIER graph})

\begin{risunok}[ht]
    \centering
    \begin{tikzpicture}
        \node
        [style={rectangle, draw=green!60, fill=green!5, very thick}]
        (Susceptible){Susceptible};
        \node
        [style={rectangle, draw=yellow!60, fill=yellow!5, very thick}]
        (Exposed)[right=of Susceptible]{Exposed};
        \node
        [style={rectangle, draw=red!60, fill=red!5, very thick}]
        (Infectious)[right=of Exposed]{Infectious};
        \node
        [style={rectangle, draw=blue!60, fill=blue!5, very thick}]
        (Removed)[right=of Infectious]{Removed};
        \draw[->, style={very thick}] (Susceptible.east)
        to node[above] {$ \frac{\beta}{N} $} (Exposed.west);
        \draw[->, style={very thick}] (Exposed.east)
        to node[above] {$ a $} (Infectious.west);
        \draw[->, style={very thick}] (Infectious.east)
        to node[above] {$  \gamma $} (Removed.west);
    \end{tikzpicture}
    \vspace{0.5cm}
    \caption{Схема роботи SIER моделі}
    \label{SIER graph}
\end{risunok}

У вихідну систему додається додаткове рівняння (див рівняння \ref{SEIR})

\begin{equation}
    \begin{cases}
        \frac{dS}{dt} = - \frac{\beta}{N}SI    \\
        \frac{dE}{dt} = \frac{\beta}{N}SI - aE \\
        \frac{dI}{dt} = aE - \gamma I          \\
        \frac{dR}{dt} = \gamma I
    \end{cases}
\end{equation}
\label{SEIR}

E(t) функція кількості заражених, коефіцієнт а відповідає оберненому середньому періоду переходу людини від зараженої, до повноцінного інфікованого.
Так само населення залишається незмінним: $ N = S + E + I + R $.

Подіюні прийоми застосовують для більш точного моделювання відповідно до
протікання хвороби - часто є певний період під час якого інфікований, 
ще не заражає оточуючих або заражає з меншим шансом, більш тісним контаком, 
або період до проявів перших симптомів і початком домашньої ізоляції,
носіння маски. 
Все це впливає на парамтри та рівняння моделі і на точність резульата. 
Буде складніше підібрати вірні параметри, але це може позетивно вплинути на точність.

\section{SEI модель}

Іноді моделі не ускладнюютть, а спрощують. 
Наприклад, якщо для хвороби виробити імунітет навічно не можливо, 
як от нежить, одужаних розглядати немає сенсу. 
Таким чином схема спрощується(див рис. \ref{SEI graph})

\begin{risunok}[ht]
    \centering
    \begin{tikzpicture}
        \node
        [style={rectangle, draw=green!60, fill=green!5, very thick}]
        (Susceptible){Susceptible}; 
        \node
        [style={rectangle, draw=red!60, fill=red!5, very thick}]
        (Infectious)[right=of Susceptible]{Infectious};
        \node
        [style={rectangle, draw=blue!60, fill=blue!5, very thick}]
        (Removed)[right=of Infectious]{Removed};
        \draw[->, style={very thick}] (Susceptible.east)
        to node[above] {$ \frac{\beta}{N} $} (Infectious.west);
        \draw[->, style={very thick}] (Infectious.east)
        to node[above] {$  a $} (Removed.west);
        \draw[->, style={very thick}] (Removed.north)
        .. controls +(up:10mm) and +(up:10mm) .. 
        node[above] {$ \gamma $}
        (Susceptible.north);
    \end{tikzpicture}
    \vspace{0.5cm}
    \caption{Схема роботи SEI моделі}
    \label{SEI graph}
\end{risunok}

Отже і система буде складатись з меншої кількості рівняннь. (див рівняння \ref{SEIS})

\begin{equation}
    \begin{cases}
        \frac{dS}{dt} = - \frac{\beta}{N}SI + \gamma I \\
        \frac{dE}{dt} = \frac{\beta}{N}SI - aE         \\
        \frac{dI}{dt} = aE - \gamma I                  \\
    \end{cases}
\end{equation}
\label{SEIS}


\section{SEIPR}

\begin{equation}
    \left\{\begin{array}{l}
    \frac{d S}{d t}=-\beta \frac{I}{N} S-\beta^{\prime} \frac{P}{N} S, \\
    \frac{d E}{d t}=\beta \frac{I}{N} S+\beta^{\prime} \frac{P}{N} S-\kappa E, \\
    \frac{d I}{d t}=\kappa \rho_1 E-\gamma_i I, \\
    \frac{d P}{d t}=\kappa (1 - \rho_1) E-\gamma_i P, \\
    \frac{d R}{d t}=\gamma_i(I+P)\\
    \end{array}\right.
\end{equation}
\label{SEIPR}




\section{SEIAPR}
\begin{equation}
    \left\{\begin{array}{l}
    \frac{d S}{d t}=-\beta \frac{I}{N} S-\beta^{\prime} \frac{P}{N} S, \\
    \frac{d E}{d t}=\beta \frac{I}{N} S+l \beta \frac{H}{N} S+\beta^{\prime} \frac{P}{N} S-\kappa E, \\
    \frac{d I}{d t}=\kappa \rho_1 E-\gamma_i I, \\
    \frac{d P}{d t}=\kappa \rho_2 E-\gamma_i P, \\
    \frac{d A}{d t}=\kappa\left(1-\rho_1-\rho_2\right) E, \\
    \frac{d R}{d t}=\gamma_i(I+P)\\
    \end{array}\right.
\end{equation}
\label{SEIAPR}


\section{Смертність}

Якщо зробити припущення, що людина у кожній групі може померти з ймовірністю $\mu$, то кожна група буде зменшуватися на добуток $\mu$ та популяцію цієї групи. 
В таких моделях також додають до розгляду новонароджених. Якщо припустити, 
що у середньому народжуваність за момент часу $dt$ дорівнює $\Lambda$, 
одержемо нуву систему рівняннь. (див рівняння \ref{SIR with mortality})  


\begin{equation}
    \begin{cases}
        \frac{dS}{dt} = \Lambda -  \mu S - \frac{\beta}{N}SI  \\
        \frac{dI}{dt} = \frac{\beta}{N}SI - \gamma I -  \mu I \\
        \frac{dR}{dt} = \gamma I -  \mu R
    \end{cases}
\end{equation}
\label{SIR with mortality}


У даній моделі населення вже буде змінюватись. (див рівняння \ref{Population})

\begin{equation}
    \frac{dN(t)}{dt} = \Lambda - \mu N(t) = \Lambda - \mu (S(t) + I(t) + R(t))
\end{equation}
\label{Population}


Іноді для спрощення встановлюють $ \Lambda = \mu $ - за таких значень 
населення не змінюється, проте трохи збільшується потенціал інфікування (з приростом нового покоління у категорію здорових)




\section{SIR-подібні моделі з вакцинацією}


\begin{equation}
    \begin{cases}
        \frac{dS}{dt} = v N (1 - P) - \frac{\beta}{N}SI          \\
        \frac{dI}{dt} = \frac{\beta}{N}SI - \gamma I \\
        \frac{dR}{dt} = \gamma I \\
        \frac{dV}{dt} = v N P
    \end{cases}
\end{equation}
\label{SIR vaccinaton}
