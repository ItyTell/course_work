\chapter{SIR модель}
В першому розгляді розглянемо SIR-подібні моделі. Почнемо з базової 
SIR моделі і поступово будемо ускладнювати модель шляхом додавання
нових груп, що дозволить поступово наблизити модель до реалій протікання
хвороби.

\section{SIR модель}
\par Одною з найпростіших моделей є SIR модель. 
В ній всі люди розглядаються у вигляді трьох груп: здорові (susceptible), хворі (infectious) та ті, що одужали (removed). 
Форсування людей між цими групами відбувається зі здорових у хворі 
і з хворих до одужаних.\cite{weiss_sir_2006}(див рис. \ref{SIR graph})  

\vspace{1cm}
\begin{risunok}[ht]
    \centering
    \begin{tikzpicture}
        \node
        [style={rectangle, draw=green!60, fill=green!5, very thick}]
        (Susceptible){Susceptible}; 
        \node
        [style={rectangle, draw=red!60, fill=red!5, very thick}]
        (Infectious)[right=of Susceptible]{Infectious};
        \node
        [style={rectangle, draw=blue!60, fill=blue!5, very thick}]
        (Removed)[right=of Infectious]{Removed};
        \draw[->, style={very thick}] (Susceptible.east)
        to node[above] {$ \frac{\beta}{N} $} (Infectious.west);
        \draw[->, style={very thick}] (Infectious.east)
        to node[above] {$  \gamma $} (Removed.west);
    \end{tikzpicture}
    \vspace{0.5cm}
    \caption{SIR model}
    \label{SIR graph}
\end{risunok}

Зв’язок між цими групами записують за допомогою системи диференціальних рівняннь 

\begin{equation*}
    \begin{cases}
        \frac{dS}{dt} = - \frac{\beta}{N}SI          \\
        \frac{dI}{dt} = \frac{\beta}{N}SI - \gamma I \\
        \frac{dR}{dt} = \gamma I
    \end{cases}
\end{equation*}


Перше рівняння описує,
що кількість здорових зменшується на величину 
пропорційну добутку хворих на здорових - кількість можливих контактів. 
Друге рівняння показує, що кількість хворих зростає на ту саму величну і зменшується на величину пропорційну до кількості хворих (кількість одужаних). 
Третє рівняння зібльшується на ту саму кількість одужаних.
\par 
Додавши всі рівняння системи отримаємо - $ \frac{dS}{dt} + \frac{dI}{dt} + \frac{dR}{dt} = 0 $. 
Тож населення у нашій моделі не змінюєтся з часом: 
$N = S + I + R = const$.\cite{salimipour_sir_2023}


\section{SEIR модель}

Часто хвороба характеризується не лише станами - хворий, здоровий, перехворівший.
Для подібних випадків додають додаткові стани.
Наприклад, якщо додати у SIR модель ще одну групу «заражені, але ще не взмозі інфікувати інших» то вийде SEIR модель. 
Тоді у порівнянні з SIR моделлю у схему додасться проміжний етап. (див рис. \ref{SIER graph})

\begin{risunok}[ht]
    \centering
    \begin{tikzpicture}
        \node
        [style={rectangle, draw=green!60, fill=green!5, very thick}]
        (Susceptible){Susceptible};
        \node
        [style={rectangle, draw=yellow!60, fill=yellow!5, very thick}]
        (Exposed)[right=of Susceptible]{Exposed};
        \node
        [style={rectangle, draw=red!60, fill=red!5, very thick}]
        (Infectious)[right=of Exposed]{Infectious};
        \node
        [style={rectangle, draw=blue!60, fill=blue!5, very thick}]
        (Removed)[right=of Infectious]{Removed};
        \draw[->, style={very thick}] (Susceptible.east)
        to node[above] {$ \frac{\beta}{N} $} (Exposed.west);
        \draw[->, style={very thick}] (Exposed.east)
        to node[above] {$ a $} (Infectious.west);
        \draw[->, style={very thick}] (Infectious.east)
        to node[above] {$  \gamma $} (Removed.west);
    \end{tikzpicture}
    \vspace{0.5cm}
    \caption{Схема роботи SIER моделі}
    \label{SIER graph}
\end{risunok}

У вихідну систему додається додаткове рівняння

\begin{equation*}
    \begin{cases}
        \frac{dS}{dt} = - \frac{\beta}{N}SI    \\
        \frac{dE}{dt} = \frac{\beta}{N}SI - aE \\
        \frac{dI}{dt} = aE - \gamma I          \\
        \frac{dR}{dt} = \gamma I
    \end{cases}
\end{equation*}

E(t) функція кількості заражених, коефіцієнт а відповідає оберненому середньому періоду переходу людини від зараженої, до повноцінного інфікованого.
Так само населення залишається незмінним: $ N = S + E + I + R $.

Подібні прийоми застосовують для більш точного моделювання відповідно до
протікання хвороби - часто є певний період під час якого інфікований, 
ще не заражає оточуючих або заражає з меншим шансом, більш тісним контаком, 
або період до проявів перших симптомів і початком домашньої ізоляції,
носіння маски. 
Це впливає на параметри та рівняння моделі і на точність резульата. 
Для даної моделі буде складніше підібрати вірні параметри, 
але це може позетивно вплинути на точність. \cite{rahimi_review_2021}

\section{SEI модель}

Іноді моделі не ускладнюютть, а спрощують. 
Наприклад, якщо для хвороби виробити імунітет навічно не можливо, 
як от нежить, одужаних розглядати немає сенсу.\cite{kim_asymptotic_2008} 
Таким чином схема спрощується(див рис. \ref{SEI graph})

\begin{risunok}[ht]
    \centering
    \begin{tikzpicture}
        \node
        [style={rectangle, draw=green!60, fill=green!5, very thick}]
        (Susceptible){Susceptible}; 
        \node
        [style={rectangle, draw=yellow!60, fill=yellow!5, very thick}]
        (Exposed)[right=of Susceptible]{Exposed};
        \node
        [style={rectangle, draw=red!60, fill=red!5, very thick}]
        (Infectious)[right=of Exposed]{Infectious};
        \draw[->, style={very thick}] (Susceptible.east)
        to node[above] {$ \frac{\beta}{N} $} (Exposed.west);
        \draw[->, style={very thick}] (Exposed.east)
        to node[above] {$  a $} (Infectious.west);
        \draw[->, style={very thick}] (Infectious.north)
        .. controls +(up:10mm) and +(up:10mm) .. 
        node[above] {$ \gamma $}
        (Susceptible.north);
    \end{tikzpicture}
    \vspace{0.5cm}
    \caption{Схема роботи SEI моделі}
    \label{SEI graph}
\end{risunok}

Отже і система буде складатись з меншої кількості рівняннь.

\begin{equation*}
    \begin{cases}
        \frac{dS}{dt} = - \frac{\beta}{N}SI + \gamma I \\
        \frac{dE}{dt} = \frac{\beta}{N}SI - aE         \\
        \frac{dI}{dt} = aE - \gamma I                  \\
    \end{cases}
\end{equation*}

\section{SEIPR}

Додамо нову групу інфікованих, яка заражає здорових людей з більшою 
чи меншою ймовірністю ніж звичайні інфіковані. 

\begin{risunok}[ht]
    \centering
    \begin{tikzpicture}
        \node at (-1, 0)
        [style={rectangle, draw=green!60, fill=green!5, very thick}]
        (Susceptible){Susceptible}; 
        \node
        [style={rectangle, draw=yellow!60, fill=yellow!5, very thick}]
        (Exposed)[right=of Susceptible, xshift = 5mm]{Exposed};
        \node
        [style={rectangle, draw=red!60, fill=red!5, very thick}]
        (Infectious)[right=of Exposed, xshift = 21mm, yshift= 18mm]
        {Infectious};
        \node
        [style={rectangle, draw=red!80, fill=red!8, very thick}]
        (Super-spreaders)[right=of Exposed, xshift = 15mm, yshift= -18mm]
        {Super-spreaders};
        \node 
        [style={rectangle, draw=blue!60, fill=red!5, very thick}]
        (Removed)[above=of Super-spreaders]{Removed};
        \draw[->, style={very thick}] (Susceptible.east)
        to node[above] {$ \frac{\beta}{N} + \frac{\beta_1}{N} $} 
        (Exposed.west);
        \draw[->, style={very thick}] (Exposed.east)
        to node[above] {$  a p_1 $} (Infectious.west);
        \draw[->, style={very thick}] (Exposed.east)
        to node[below, xshift = -7mm] {$  a (1 - p_1) $} 
        (Super-spreaders.west);
        \draw[->, style={very thick}] (Infectious.south)
        to node[left] {$  \gamma_i $} 
        (Removed.north);
        \draw[->, style={very thick}] (Super-spreaders.north)
        to node[left] {$  \gamma_p $} 
        (Removed.south);
    \end{tikzpicture}
    \vspace{0.5cm}
    \caption{Схема роботи SEIPR моделі}
    \label{SEIPR graph}
\end{risunok}



Тоді кожен заражений може стати або звичайним інфікованим з 
ймовірністю $p_1$ або іншим інфікованим з ймовірністю 
$1 - p_1$. Параметр, що відповідає за інфікування 
інших інфікованих $\beta_1$. Для спрощення моделі часто припускають, 
що ці дві групи інфікованих мають однакову ймовірність одужати і беруть 
однакові коефіцієнти $\gamma_i = \gamma_p = \gamma$, але загальна 
система виглядає так: 


\begin{equation*}
    \left\{\begin{array}{l}
    \frac{d S}{d t}=-\beta \frac{I}{N} S-\beta^{\prime} \frac{P}{N} S, \\
    \frac{d E}{d t}=\beta \frac{I}{N} S+\beta^{\prime} \frac{P}{N} S-\kappa E, \\
    \frac{d I}{d t}=\kappa \rho_1 E-\gamma_i I, \\
    \frac{d P}{d t}=\kappa (1 - \rho_1) E-\gamma_p P, \\
    \frac{d R}{d t}=\gamma_i I + \gamma_p P\\
    \end{array}\right.
\end{equation*}



\section{SEIAPR}


Додамо групу людей, які заражені проте не мають ніяких симптомів і не 
можуть заражати інших. 

\begin{risunok}[ht]
    \centering
    \begin{tikzpicture}
        \node at (-1, 0)
        [style={rectangle, draw=green!60, fill=green!5, very thick}]
        (Susceptible){Susceptible}; 
        \node
        [style={rectangle, draw=yellow!60, fill=yellow!5, very thick}]
        (Exposed)[right=of Susceptible, xshift = 5mm]{Exposed};
        \node
        [style={rectangle, draw=red!60, fill=red!5, very thick}]
        (Infectious)[right=of Exposed, xshift = 21mm, yshift= 18mm]
        {Infectious};
        \node 
        [style={rectangle, draw=orange!60, fill=orange!5, very thick}]
        (Asymptomatic)[below=of Exposed]
        {Asymptomatic};
        \node
        [style={rectangle, draw=red!80, fill=red!8, very thick}]
        (Super-spreaders)[right=of Exposed, xshift = 15mm, yshift= -18mm]
        {Super-spreaders};
        \node 
        [style={rectangle, draw=blue!60, fill=red!5, very thick}]
        (Removed)[above=of Super-spreaders]{Removed};
        \draw[->, style={very thick}] (Susceptible.east)
        to node[above] {$ \frac{\beta}{N} + \frac{\beta_1}{N} $} 
        (Exposed.west);
        \draw[->, style={very thick}] (Exposed.east)
        to node[above] {$  a p_1 $} (Infectious.west);
        \draw[->, style={very thick}] (Exposed.east)
        to node[below, xshift = -2mm] {$  a p_2 $} 
        (Super-spreaders.west);
        \draw[->, style={very thick}] (Infectious.south)
        to node[left] {$  \gamma_i $} 
        (Removed.north);
        \draw[->, style={very thick}] (Super-spreaders.north)
        to node[left] {$  \gamma_p $} 
        (Removed.south);
        \draw[->, style={very thick}] (Exposed.south)
        to node[left] {$  a(1 - p_1 - p_2) $} 
        (Asymptomatic.north);
    \end{tikzpicture}
    \vspace{0.5cm}
    \caption{Схема роботи SEIAPR моделі}
    \label{SEIAPR graph}
\end{risunok}



В данному випадку нас не цікавить чи вилікуються вони
чи ні адже вони впливають на систему так само як і виліковані. Тоді кожен 
заражений з категорії $E$ з ймовірністю $p_1$ зможе стати зараженим $I$ 
з ймовірністю $p_2$ може стати зараженим $P$ і з ймовірністю $1 - p_1 - p_2$
може стати зараженим $A$.\cite{palka_using_2022} 
\begin{equation*}
    \left\{\begin{array}{l}
    \frac{d S}{d t}=-\beta \frac{I}{N} S-\beta^{\prime} \frac{P}{N} S, \\
    \frac{d E}{d t}=\beta \frac{I}{N} S +\beta^{\prime} \frac{P}{N} S- aE, \\
    \frac{d I}{d t}= a \rho_1 E-\gamma_i I, \\
    \frac{d P}{d t}= a \rho_2 E-\gamma_p P, \\
    \frac{d A}{d t}= a\left(1-\rho_1-\rho_2\right) E, \\
    \frac{d R}{d t}=\gamma_i I + \gamma_p P\\
    \end{array}\right.
\end{equation*}

\section{SEIPHRD}


Остання модель була взята з статті прогнозування захворюваності у 
Вухані.\cite{ndairou_mathematical_2020} 
Тут додані дві нові групи 
H- госпіталізовані та F- летальні випадки. Це допоможе нам конфігурувати 
моделі адже подібні дані дуже часто зберігаються і вони є більш точними 
ніж кількість інфікованих. 

\begin{risunok}[ht]
    \centering
    \begin{tikzpicture}
        \node at (-1, 0)
        [style={rectangle, draw=green!60, fill=green!5, very thick}]
        (Susceptible){Susceptible}; 
        \node
        [style={rectangle, draw=yellow!60, fill=yellow!5, very thick}]
        (Exposed)[right=of Susceptible, xshift = 5mm]{Exposed};
        \node
        [style={rectangle, draw=red!60, fill=red!5, very thick}]
        (Infectious)[right=of Exposed, xshift = 21mm, yshift= 18mm]
        {Infectious};
        \node 
        [style={rectangle, draw=orange!60, fill=orange!5, very thick}]
        (Asymptomatic)[below=of Exposed]
        {Asymptomatic};
        \node
        [style={rectangle, draw=red!80, fill=red!8, very thick}]
        (Super-spreaders)[right=of Exposed, xshift = 15mm, yshift= -18mm]
        {Super-spreaders};
        \node 
        [style={rectangle, draw=blue!60, fill=red!5, very thick}]
        (Removed)[above=of Super-spreaders]{Removed};

        \node
        [style={rectangle, draw=green!60, fill=green!5, very thick}]
        (Hospitalize)[right=of Infectious, xshift = 6mm]
        {Hospitalize};

        \node
        [style={rectangle, draw=black!60, fill=black!5, very thick}]
        (Fatal)[right=of Super-spreaders, xshift= 12mm]{Fatal};

        \draw[->, style={very thick}] (Susceptible.east)
        to node[above] {$ \frac{\beta}{N} + \frac{\beta_1}{N} $} 
        (Exposed.west);
        \draw[->, style={very thick}] (Exposed.east)
        to node[above] {$  a p_1 $} (Infectious.west);
        \draw[->, style={very thick}] (Exposed.east)
        to node[below, xshift = -2mm] {$  a p_2 $} 
        (Super-spreaders.west);
        \draw[->, style={very thick}] (Infectious.south)
        to node[left] {$  \gamma_i $} 
        (Removed.north);
        \draw[->, style={very thick}] (Super-spreaders.north)
        to node[left] {$  \gamma_p $} 
        (Removed.south);
        \draw[->, style={very thick}] (Exposed.south)
        to node[left] {$  a(1 - p_1 - p_2) $} 
        (Asymptomatic.north);

        \draw[->, style={very thick}] (Infectious.east)
        .. controls +(up:5mm) and +(up:5mm) .. 
        node[above] {$  \gamma_a $} 
        (Hospitalize.west);

        \draw[->, style={very thick}] (Super-spreaders.north)
        to node[below, xshift =-2mm, yshift = -2mm] {$ \gamma_a $}
        (Hospitalize.south);

        \draw[->, style={very thick}] (Hospitalize.west)
        to node[above, xshift = -7mm, yshift = -7mm] {$  \gamma_r $} 
        (Removed.east);

        \draw[->, style={very thick}] (Infectious.east)
        to node[below, xshift = 5mm, yshift = -7mm] {$  \sigma_i $} 
        (Fatal.west);

        \draw[->, style={very thick}] (Super-spreaders.east)
        to node[below] {$  \sigma_p $} 
        (Fatal.west);

        \draw[->, style={very thick}] (Hospitalize.south)
        to node[above, xshift= 4mm] {$  \sigma_h $} 
        (Fatal.north);

    \end{tikzpicture}
    \vspace{0.5cm}
    \caption{Схема роботи SEIAPHRF моделі}
    \label{SEIAPHRF graph}
\end{risunok}




За смерть різниї груп відповідають параметри 
$\sigma_i, \sigma_p, \sigma_h$, а за госпіталізацію параметр
$\gamma_a$. 

\begin{equation*}
    \left\{\begin{array}{l}
    \frac{d S}{d t}=-\beta \frac{I}{N} S-l \beta \frac{H}{N} S-\beta^{\prime} \frac{P}{N} S, \\
    \frac{d E}{d t}=\beta \frac{I}{N} S+l \beta \frac{H}{N} S+\beta^{\prime} \frac{P}{N} S- a E, \\
    \frac{d I}{d t}= a \rho_1 E-\left(\gamma_a+\gamma_i\right) I-\delta_i I, \\
    \frac{d P}{d t}= a\rho_2 E-\left(\gamma_a+\gamma_i\right) P-\delta_p P, \\
    \frac{d A}{d t}= a\left(1-\rho_1-\rho_2\right) E, \\
    \frac{d H}{d t}=\gamma_a(I+P)-\gamma_r H-\delta_h H, \\
    \frac{d R}{d t}=\gamma_i(I+P)+\gamma_r H, \\
    \frac{d F}{d t}=\delta_i I+\delta_p P+\delta_h H,
    \end{array}\right.
\end{equation*}
