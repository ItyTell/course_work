
\chapter{Чисельні методи}


У другому розділі розглянемо чисельні методи для розв'язання 
данних сисетем диф рівняннь та для визначення оптимальних параметрів, що 
якнайкраще симулюють розвиток захворювансті.

\section{Методи вирішення системи диф рівняннь}

\subsection{Метод Ейлера}


Метод Ейлера найбільш базовий чисельний метод для розв'язання нелінійних
систем диференціальних рівняннь, що застосувується при відомих початкових
значеннях. 
В данному випадку нам відома кількість хворих на початку епідемії.
Це частковий випадок методів Рунге Кутті. 
Загалом метод можна описати формулою $dS \approx S_{n+1} - S_n$.
Зазвичай береться певний крок $dt$ і дифиренціальне рівняння 
апроксимується відносно нього:

$$\frac{dS}{dt} = f(t, S) \Rightarrow 
S_{n+1} = S_n + f(t_n, S_n)$$

Ітерації починаються при відомому $S_0$. Метод є не стабільним при великому 
кроці $dt$.

Як приклад візьмемо SIR модель при розв'язанні моделі отримаємо:

$$
\begin{cases}
    S_{n+1} = S_n - \frac{\beta}{N} S_n I_n dt \\
    I_{n+1} = I_n + \frac{\beta}{N} S_n I_n dt - \gamma I_n dt \\
    R_{n+1} = R_n + \gamma I_n dt \\
    S_0 = N - 1, I_0 = 1, R_0 = 0
\end{cases}
$$

\pagebreak

Функція написана на С для отримання данних з моделі відносно її параметрів 
кроку та початкових умов. 
\lstinputlisting[style=CStyle]{main.c}

Мова програмування С була обрана для написання 
цієї функції завдяки її швидкості виконання та відносної простоти 
написанння коду. Потреба у високій швидкості виконання обумовлена потребами 
задля швидкого знахордження оптимальних параметрів - через складну 
залежність від параметрів моделі використати градієнтні методи не вийде, а
не градієнтні методи можуть багато разів викликати данну функцію на кожній 
ітерації. 


Функції, що розв'язують інші моделі відрізняються лише кількістю масивів та 
основним циклом, загальна структура функцій однакова. 
Час виконання програми такий то. 


\subsection{Метод Рунге Кутті}

\section{Методи підбору параметрів}


\subsection{Метод рою часток}

