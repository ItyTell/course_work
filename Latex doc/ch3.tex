\chapter{Оцінка і порівняння моделей}


У ході курсової роботи для оцінки та візуалізації моделей була створена 
програма, яка дозволяє користувачу регулювати параметри та одразу бачити, 
як вони впливають на прогнози. Також реалізована кнопка, яка починає 
оптимізацїю початкових параметрів, налаштованих користувачем, методом рою 
часток (одна з часток ініціалізована початковими параметрами). Також 
є опція зберегти параметри у форматі json. 
Основна програма написана на мові програмування Python, функції поделювання 
написані на C задля збільшення швидкості оптимізації.


\section{Справжні дані та оцінка ефективності}

Прогнозувати будемо розвиток коронавірусу у Люксимбургу через те, що країна 
досить не велика та має середню густоту населення. 
Підбирати параметри будемо за перші 70 днів епідемії і відповідний прогноз 
будемо робити на наступні два тижні. 

\section{SIR}
\section{SEIR}
\section{SEIPR}
\section{SEIAPR}
\section{SEIAPHR}
\section{SEIAPHRD}
\section{Порівняння моделей та результати}