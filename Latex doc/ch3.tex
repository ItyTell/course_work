\chapter{Оцінка і порівняння моделей}


У ході курсової роботи для оцінки та візуалізації моделей була створена 
програма, яка дозволяє користувачу регулювати параметри та одразу бачити, 
як вони впливають на прогнози. Також реалізована кнопка, яка починає 
оптимізацїю початкових параметрів, налаштованих користувачем, методом рою 
часток (одна з часток ініціалізована початковими параметрами). Також 
є опція зберегти параметри у форматі json. 
Основна програма написана на мові програмування Python, функції поделювання 
написані на C задля збільшення швидкості оптимізації.


\section{Справжні дані та оцінка ефективності}

Прогнозувати будемо розвиток коронавірусу у Люксимбургу через те, що країна 
досить не велика та має середню густоту населення. 
Підбирати параметри будемо за перші 50 днів епідемії і відповідний прогноз 
будемо робити на наступні два тижні. 

\section{SIR}


SIR модель погано моделює навіть ту частину графіка, яку ми використали для 
налаштування її параметрів. Середня різниця між графіками на відрізку 
прогнозування (20 днів) має порядк сотень людей. 
Сумма квадратів відстаней між графіками у кожній цілій точці 
складає 4906780.


\begin{figure}[H]
    \centering
    \includegraphics[scale=0.5]{SIR365.png}
    \caption{Прогноз SIR моделі після 100 днів}
    \label{fig:plot1}
\end{figure}


\section{SEIR}


З SEIR моделлю вже можливо досить непогано апроксимувати графік, проте 
прогноз і реальні дані мають зовсім різну динаміку, порядок похибки - 
тисячі, отже дану модель не можна використовувати для пронозів. 
Відстань між графіками 36695095347.


\begin{figure}[H]
    \centering
    \includegraphics[scale=0.5]{SEIR365.png}
    \caption{Прогноз SEIR моделі після 365 днів}
    \label{fig:plot2}
\end{figure}
\section{SEIPR}

\begin{figure}[H]
    \centering
    \includegraphics[scale=0.5]{SEIPR365.png}
    \caption{Прогноз SIR моделі після 100 днів}
    \label{fig:plot3}
\end{figure}
\section{SEIAPR}
\begin{figure}[H]
    \centering
    \includegraphics[scale=0.5]{SEIAPR365.png}
    \caption{Прогноз SIR моделі після 100 днів}
    \label{fig:plot4}
\end{figure}
\section{SEIAPHRD}
\begin{figure}[H]
    \centering
    \includegraphics[scale=0.5]{model1_365.png}
    \caption{Прогноз SIR моделі після 100 днів}
    \label{fig:plot5}
\end{figure}
\section{Порівняння моделей та результати}