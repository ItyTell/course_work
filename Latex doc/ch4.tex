\chapter*{Висновки}


Підсумовуючи нашу роботу, ми встановили, що всі розглянуті моделі
моделювання розвитку епідемії, мають проблеми, особливо при моделюванні 
початку епідемії та в місцях різкої зміни динаміки розвитку захворюваності.
Ці проблеми можуть виникати через неповну або неточну інформацію про
початкові умови, особливості поширення та виявлення захворювання на початку
епідемії та появи різних заходів захисту населення.

Незважаючи на проблеми, які виникають, ми також виявили, що деякі 
моделі спроможні дати досить точний короткостроковий прогноз. 
Вони можуть бути використані для оцінки поточного стану епідемії, 
планування необхідних заходів для контролю та протидії захворюванню та 
оцінки заходів захисту населення.
Це може бути особливо корисним для стратегічного планування та 
прийняття рішень у галузі громадського здоров'я.

Чисельні методи виявляються корисними інструментами для дослідження
розвитку епідемії та прогнозування короткострокових сценаріїв. 
Вони можуть допомогти нам краще розуміти та управляти епідеміологічними
ситуаціями, сприяючи здоров'ю та безпеці громадськості.