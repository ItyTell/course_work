\chapter*{ВСТУП}

\textbf{Оцінка сучасного стану об’єкта дослідження}


У рамках даної курсової роботи, нашим об'єктом дослідження є розвиток епідемії. Для отримання глибокого розуміння даного об'єкта, ми здійснюємо оцінку його сучасного стану. Ця оцінка базується на дослідженні актуальних даних та використанні чисельних методів.

Ми вивчаємо важливі фактори, які впливають на поширення епідемії, такі як швидкість передачі, рівень вакцинації, ефективність заходів контролю та інші. Це дозволяє нам отримати більш точну картину та зробити адекватні прогнози.

Оцінка сучасного стану об'єкта дослідження також включає використання чисельних методів. Ми використовуємо математичні моделі для моделювання поширення епідемії на основі наявних даних. Ці моделі дозволяють нам розрахувати рівень інфікування, динаміку поширення та імітацію різних сценаріїв.

Отримані результати оцінки сучасного стану об'єкта дослідження стануть основою для подальшого моделювання розвитку епідемії та визначення ефективних стратегій контролю та мінімізації поширення захворювання.


\textbf{Актуальність роботи та підстави для її виконання.}

Наша курсова робота, що присвячена моделюванню розвитку епідемії з використанням чисельних методів, є дуже актуальною в контексті сучасних глобальних викликів у сфері громадського здоров'я. Ось деякі підстави, які обгрунтовують актуальність нашої роботи:

    Пандемія COVID-19: Світ стикається з найпоширенішою епідемією за останнє десятиліття, яка має глибокий вплив на життя людей та економіку. Розуміння та моделювання розвитку епідемій, таких як COVID-19, є надзвичайно важливими для розробки ефективних стратегій контролю та прогнозування їхнього впливу.

    Потреба у прогнозуванні: Розуміння майбутнього розвитку епідемій дозволяє приймати обґрунтовані рішення щодо адаптації систем охорони здоров'я, вакцинації, медичного обладнання та інших аспектів. Моделювання розвитку епідемії з використанням чисельних методів надає можливість прогнозувати варіанти розвитку подій та оцінювати ефективність різних стратегій.

    Розвиток наукових методів: Чисельні методи використовуються в багатьох наукових дисциплінах для моделювання складних процесів. Застосування цих методів до дослідження епідеміологічних моделей дозволяє отримати детальніші та точніші результати, що сприяє покращенню нашого розуміння поширення епідемій.

    Перспектива майбутніх епідемій: Історія свідчить про те, що епідемії та пандемії є неминучою частиною нашого світу. Моделювання розвитку епідемій з використанням чисельних методів допоможе нам готуватися до майбутніх викликів і впроваджувати належні заходи передбачення, мінімізації поширення та зменшення наслідків епідемій.

Отже, наша робота є актуальною та важливою, оскільки вона спрямована на дослідження моделей розвиту епідемій. Вона відповідає сучасним викликам у галузі громадського здоров'я та може стати цінним інструментом для прогнозування та управління епідеміологічними заходами.

\textbf{Мета й завдання роботи. Метою є}

Метою роботи є дослідження різних моделей розвитку епідемі за допомогою 
чисельних методів. 
Завданнями роботи є:
    -огляд основних моделей розвитку епідемій
    -підбір параметрів моделей під реальні дані
    -отримання та оцінка прогнозів моделей


\textbf{Об'єкт, методи та засоби дослідження.}

Об'єктом нашого дослідження є розвиток епідемії. Ми зосереджуємося на моделюванні та прогнозуванні поширення захворювання з використанням чисельних методів.

Для досягнення наших цілей ми використовуємо наступні методи та засоби:

\begin{enumerate}
    \item Математичне моделювання: Ми використовуємо математичні моделі для опису динаміки поширення епідемії. Ці моделі базуються на епідеміологічних принципах та інформації про характеристики захворювання. Вони дозволяють нам розрахувати рівень інфікування, швидкість поширення та імітувати різні сценарії розвитку епідемії.

    \item Чисельні методи: Для розв'язання математичних моделей ми використовуємо чисельні методи. Ці методи дозволяють нам апроксимувати розв'язок диференціальних рівнянь та виконувати чисельні симуляції розвитку епідемії. Вони забезпечують точність та ефективність обчислень.

    \item Програмування на Python та C: Ми використовуємо мови програмування Python та C для реалізації наших чисельних методів та моделей. Python є широко використовуваною мовою програмування у сфері наукових обчислень, зокрема для моделювання епідемій. C є мовою програмування, що надає високу швидкодію, і може бути використана для оптимізації обчислювальних процесів.

    \item Аналіз даних: Для оцінки сучасного стану об'єкта дослідження ми здійснюємо аналіз актуальних даних щодо поширення епідемії. Ми використовуємо статистичні методи та інструменти для обробки та визначення закономірностей у даних.

    \item Комп'ютерне моделювання: Ми використовуємо комп'ютерні моделі та інструменти для виконання чисельних симуляцій та візуалізації результатів. Це дозволяє нам вивчати розвиток епідемії на основі введених параметрів та проводити аналіз різних сценаріїв.

\end{enumerate}
Загалом, наш дослідницький підхід поєднує математичне моделювання, чисельні методи та програмування, зокрема використання мов Python та C, для досягнення наших цілей у моделюванні розвитку епідемії.

\textbf{Можливі сфери застосування.}

За допомогою подібних дій описаних у даній роботі можна оцінювати моделі 
розвитку епідемій і таким чином вибирати найкращі ідеї. Завдякт цьому 


